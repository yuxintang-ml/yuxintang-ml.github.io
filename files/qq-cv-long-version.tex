%____________________________________________________________________
% @brief    LaTeX2e Resume for  Qingqing Huang

\documentclass[margin,line,11pt]{resume}
%% % PDF SETUP
%% % ---- FILL IN HERE THE DOC TITLE AND AUTHOR
\usepackage{textcomp}
\usepackage{xcolor}
\definecolor{navy}{HTML}{2F729C}

\begin{document}
%______________________________________________________________
% \vspace{-3.5mm}\name
{\Large \textbf{Qingqing Huang}}\hspace{52mm}
{\footnotesize +1 617.909.3319 $\vert$ qqh@mit.edu  $\vert$ http://qingqinghuang.github.io}
\begin{resume}


% %____________________________________________________________________
% % @brief    LaTeX2e Resume for Qingqing Huang
% \documentclass[margin,line,11pt]{resume}
% %% % PDF SETUP
% %% % ---- FILL IN HERE THE DOC TITLE AND AUTHOR
% \usepackage{textcomp}
% \usepackage{xcolor}
% \definecolor{navy}{HTML}{2F729C}

% \begin{document}
% %______________________________________________________________
% {\Large \textbf{Qingqing Huang}}\hspace{58mm} {\footnotesize qqh@mit.edu $\vert$
%   617.909.3319 $\vert$ http://qingqinghuang.github.io}
% \begin{resume}


  %__________________________________________________________
  % Contact Information
  \section{\mysidestyle Affiliation}
  Laboratory of Information and Decision Systems, \\
  Department of Electrical Engineering and Computer Sciences,\\
  Massachusetts Institute of Technology

  \vspace{-12pt}
  % ________________________________________________________
  % Education
  \section{\mysidestyle Education}
  \textbf{Massachusetts Institute of Technology (MIT)}

  \vspace{-12pt}
  \textsl{Ph.D candidate in Electrical Engineering} \hfill \textbf{June 2013 -- May 2016
    (expected)}
  \\
  {Thesis Topic}: Make impossible possible, make possible optimal, make optimal practical
  -- algorithmic statistics for learning mixture models.
  \\
  {Thesis Committee}: Munther Dahleh, Sham Kakade, Pablo Parrilo.

  \vspace{-12pt}
  \textsl{Master in Electrical Engineering} \hfill \textbf{ Sep 2011 -- May
    2013}
  \\
  {Thesis Topic}: Efficiency-Risk Tradeoffs in Electricity Markets with Dynamic Demand
  Response
  \\
  {Research Advisor}: Munther Dahleh

  \vspace{-10pt}
  \textbf{Hong Kong University of Science and Technology (HKUST)}

  \vspace{-12pt}
  \textsl{Bachelor of Engineering in Electrical Engineering} \hfill \textbf{ Sep 2006 --
    June 2011}
  \\
  \textsl{Bachelor of Business Administration in Economics} \hfill \textbf{ Sep 2006 --
    June 2011}

  %_________________________________________________________
  % Research Interests
  \section{\mysidestyle Research Interests}
  \textbf{Statistical learning theory, machine Learning, networked systems.}

  \vspace{-6pt}

  \textbf{Ph.D thesis topic:} I focus on a set of statistical learning problems regarding
  mixture models.
  \\
  Mixture models (examples include Gaussian Mixtures (GMM), topic models, and Hidden
  Markov Models (HMM)) serve to model the scenario where the underlying mechanism of each
  observed data sample belongs to a finite number of different sources.
  %
  It is a class of powerful models which finds application in a wide range of unsupervised
  learning tasks, such as speech recognition, document classification, super-resolution
  imaging, community detection, and low rank matrix recovery for recommendation tasks.

  \vspace{-8pt}

  The structural property of the distribution with the latent variable introduces
  non-convexity to the learning problem, making it much harder than the unstructured
  problems.
 %
  To this end, I ask and attempt address three questions: Can we efficiently learn the
  model parameters, assuming some non-degeneracy of the instances?  Can we achieve it with
  optimal sample complexity with fast algorithms? Can we make the learning algorithms also
  robust to model mis-specifications?
  % A more intriguing question is that, is there a unified theoretical framework to answer
  % mixture models?

  \vspace{-6pt}

  \textbf{Future research directions:} I will continue to research the broad field of
  statistical learning, in particular exploring the two aspects of \emph{robustness} and
  \emph{dynamics} of various learning questions.
  \\
  I am also particularly interested in the direction of application of statistics and
  machine learning to tackle real-life challenges in economics and sociology, in
  particular addressing the issue of sparse and noisy data in these fields, as in the
  non-asymptotic regime the conventional statistics and learning methods do not directly
  apply.


  %_____________________________________________________________________
  % Research Experience
  \section{\mysidestyle Research\\Experience }

  \textsl{Graduate researcher} at
  \textbf{Laboratory of Information and Decision Systems, MIT}
  \\
  Thesis advisor : Munther Dahleh and Sham Kakade \hfill {Sep 2011 -- Present}

  \vspace{-6pt}

  \textsl{Research Internship} at \textbf{Machine Learning Group, Microsoft Research New
    England}
  \\
  \textcolor{white}{blank} \hfill {May 2014 -- Aug 2014}
  \\
  Mentor: Sham Kakade\hfill {May 2015 -- Aug 2015}

  \vspace{-6pt}

  \textsl{Visitor} at \textbf{Big Data Lab, Baidu, Beijing}
  \\
  Host: Tong Zhang \hfill {Dec 2014}

  \vspace{-6pt}

  \textsl{Research Assistant} at \textbf{Wireless Communication Group, ECE, HKUST}
  \\
  Research advisor: Vincent K.N. Lau \hfill {June 2009 -- Jan 2011}



\newpage

\section{\mysidestyle Teaching \\ Experience}
\textbf{Department of Electrical Engineering and Computer Sciences, MIT}
\vspace{-3mm}

\textsl{Teaching Assistant} for 6.438 ``Algorithms for Inference'' \hfill {Fall 2013}
\\
It was a graduate level class with approximately 60 students with 2 TAs.
  %
TA duties include giving bi-weekly recitations, designing and administrating class
project, holding weekly office hours, assisting the instructor to prepare homework
assignments and exams.
\\
Instructor: Devavrat Shah

\textsl{Teaching Assistant} for 6.207 ``Networks'' \hfill {Spring 2014}
\\
It was an undergraduate level class with approximately 80 students with 2 TAs.
  %
TA duties include giving bi-weekly recitations, holding weekly office hours, and assisting
the instructor to prepare homework assignments and exams.

\textsl{Teaching Assistant} for 6.UAR ``Prep for Undergrad Research'' \hfill {Fall
  2015}
\\
It was an undergraduate level class with approximately 180 students with 4 TAs.
  %
The TA duties include assisting the students in their individual research projects step by
step, offering advice on project proposals and posters.



\section{\mysidestyle Extra Curricular Activities}

{\sl Publication Chair} of {\sl Graduate Student Association, MIT} \hfill Jan 2013 --
Dec 2013
\\
In charge of website maintenance, poster design for publicizing events

{\sl Co-president} of {\sl Graduate Women in Course 6 (GW6), MIT} \hfill Jan 2012 -- Dec 2012
\\
Organized extra curricular activities for graduate women in EECS department of MIT


{\sl Software Engineer Intern} at {\sl Yunzhou-Tech Company, China} \hfill June 2011 -- Aug 2011
\\
Worked on navigation algorithm improvements for unmanned surface vehicles

{\sl IT Engineering} at {\sl Kwong Wah Hospital, Hong Kong} \hfill Jan 2011 --
May 2011
\\
Developed a web-based medical image archiving system  (Student Civic Fellow Program)

{\sl Project Assistant} at {\sl Heep Hong Society, Hong Kong} \hfill June 2010 -- Oct 2010
\\
Collaborated to develop a computer-based learning package for autistic children


\section{\mysidestyle Programming}
Proficient in Matlab, C++, Python.


\section{\mysidestyle Awards}
{Xerox-MIT Fellowship}\hfill 2012\\
{Irwin Mark Jacobs and Joan Klein Jacobs Presidential Fellowship } \hfill 2011\\
% { Academic Achievement Medal }{\small(Top 1\% of graduates in HKUST)} \hfill 2011\\
%{ Lee Wan Keung Scholarships, ECE Scholarship} \hfill 2006--2011\\
{ Silver Medal in the National Physics Olympiad, China} \hfill 2005\\



\newpage
\small
\section{\mysidestyle Publications}
(Listed roughly in reverse chronological order.  Authors are ordered alphabetically for CS
theory papers)


\vspace{-5mm}
\subsection{\small\sc\textbf{Statistical learning theory}}
\vspace{-5mm}

``Recovering Structured Probability Matrices''
\textbf{H}, Sham Kakade, Wenhao Kong, Gregory Valiant,
\\
submitted to STOC 2016

``Learning Mixture of Gaussians in High dimensions ''
Rong Ge, \textbf{H}, Sham Kakade,
\\
appeared in 47th Annual Symposium on the Theory of Computing (STOC), 2015

``Super-Resolution off the Grid '' \textbf{H}, Sham Kakade
\\
appeared in the 30th Annual Conference on Neural Information Processing Systems (NIPS), 2015

``Minimal Realization Problems for Hidden Markov Models '' \textbf{H}, Rong Ge, Sham Kakade, Munther
Dahleh
\\
Journal version submitted to IEEE Transactions on Signal Processing; conference version appeared in
the 52nd Annual Allerton Conference on Communication, Control, and Computing (Allerton) , 2014.

``A Greedy Algorithm for Nonnegative Matrix and Tensor Factorization'' \textbf{H}, Tong Zhang
\\
working paper

\vspace{-5mm}
\subsection{\small\sc\textbf{Smart Grid technologies}}
\vspace{-5mm} ``Efficiency-Risk Tradeoffs in Electricity Markets with Dynamic Demand Response''
\textbf{H}, Mardavij Roozebehani, Munther Dahleh
\\
Journal version appeared in IEEE Transactions on Smart Grid (TSG), 2014
\\
Conference version appeared in IEEE Conference on Decision and Control (CDC), 2012

``Dynamic Fault Diagnosis in Power Grids Using Hidden Markov Models'' \textbf{H}, Leilai Shao, Na Li
\\
Journal version appeared in IEEE Transactions on Power System (TPS), 2015
\\
Conference version appeared in IEEE American Control Conference (ACC), 2015

\vspace{-5mm}
\subsection{\small\sc\textbf{Miscellaneous} }
\vspace{-5mm}

``H2-Based Network Volatility Measures''  \textbf{H}, Ye Yuan, Jorge Goncalves, and Munther Dahleh
\\
Conference version appeared in IEEE American Control Conference (ACC), 2014

``Queue-Aware Dynamic Clustering and Power Allocation for Network MIMO Systems via Distributed
Stochastic Learning '' Ying Cui, \textbf{H}, Vincent K.N. Lau
\\
Journal version appeared in IEEE Transactions on Signal Processing (TSP), 2011

``Delay-Optimal Orthogonal Beam forming and Power Control for MIMO system with Reduced CSI Feedback
'' \textbf{H}, Ying Cui, Vincent K.N. Lau
\\
Technical report, 2011


\section{\mysidestyle Abstract \\ of selected publications}

\textbf{``Recovering Structured Probability Matrices''}
\\
We consider the problem of recovering a matrix $B$ of size $M\times M$, which represents a
probability distribution over $M^2$ outcomes, given access to independent sample ``counts'' generated
according to the distribution $B$. How can structural properties of the underlying matrix $B$ be
leveraged to yield computationally efficient and information theoretically optimal reconstruction
algorithms? When can accurate reconstruction be accomplished in the sparse data regime?
\\
This basic problem lies at the core of a number of questions that are currently being considered by
different communities, including community detection in sparse random graphs, learning structured
probabilistic models such as topic models and hidden Markov models, as well as the efforts from the
natural language processing community to compute ``word embeddings''. Many aspects of this problem --
both in terms of learning and property testing, and on both the algorithmic and information
theoretic sides—remain open.
\\
Our results apply to the setting where the $M\times M$ probability matrix $B$ is of rank 2. We propose an
efficient algorithm that accurately recovers the underlying matrix using $\Theta(M)$ samples. The linear
sample complexity is optimal, up to constant factors, in an extremely strong sense: even testing
basic properties of the underlying matrix, such as whether it has rank 1 or 2, requires $\Omega(M)$
samples.

\textbf{``Super-Resolution off the Grid''}
\\
Super-resolution is the problem of recovering a superposition of point sources using bandlimited
measurements, which may be corrupted with noise. This signal processing problem arises in numerous
imaging problems, ranging from astronomy to biology to spectroscopy, where it is common to take
(coarse) Fourier measurements of an object.
%
Of particular interest is in obtaining estimation procedures which are robust to noise, with the
following desirable statistical and computational properties: we seek to use coarse Fourier
measurements (bounded by some \emph{cutoff frequency}); we hope to take a (quantifiably) small
number of measurements; we desire our algorithm to run quickly.
\\
Suppose we have $k$ point sources in $d$ dimensions, where the points are separated by at least
$\Delta$ from each other (in Euclidean distance). This work provides an algorithm with the following
favorable guarantees:
\\
\quad -- The algorithm uses Fourier measurements, whose frequencies are bounded by $O(1/\Delta)$ (up to log
factors). Previous algorithms require a \emph{cutoff frequency} which may be as large as
$\Omega(\sqrt{d}/\Delta)$.
\\
\quad -- The number of measurements taken by and the computational complexity of our algorithm are bounded by
a polynomial in both the number of points $k$ and the dimension $d$, with \emph{no} dependence on
the separation $\Delta$. In contrast, previous algorithms depended inverse polynomially on the
minimal separation and exponentially on the dimension for both of these quantities.
% \\
% Our estimation procedure itself is simple: we take random bandlimited measurements (as opposed to
% taking an exponential number of measurements on the hyper-grid). Furthermore, our analysis and
% algorithm are elementary (based on concentration bounds for sampling
% and the singular value
% decomposition).


\textbf{``Learning Mixture of Gaussians in High dimensions''}
\\
  Efficiently learning mixture of Gaussians is a fundamental problem in statistics and
  learning theory.  Given samples coming from a random one out of $k$ Gaussian
  distributions in $n$ dimensional space, the learning problem asks to estimate the means and the
  covariance matrices of these Gaussians.  This learning problem arises in many areas
  ranging from the natural sciences to the social sciences, and has also found many
  machine learning applications.
\\
  Unfortunately,  learning mixture of Gaussians is an information theoretically
  hard problem: in order to learn the parameters up to a reasonable accuracy, the number of samples
  required is exponential in the number of Gaussian components in the worst case.
  %\cite{moitra2010settling}.
  %
  In this work, we show that provided we are in high enough dimensions, the class of Gaussian
  mixtures is learnable in its most general form under a smoothed analysis framework, where the
  parameters are randomly perturbed from an adversarial starting point.
\\
  In particular, given samples from a mixture of Gaussians with randomly perturbed
  parameters, when $n\ge \Omega(k^2)$, we give an algorithm that learns the parameters
  with polynomial running time and using polynomial number of samples.
  \\
  The central algorithmic ideas consist of new ways to decompose the  moment tensor of
  the Gaussian mixture by exploiting its structural properties.
%
  The symmetries of this tensor are derived from the combinatorial structure of higher
  order moments of Gaussian distributions (sometimes referred to as Isserlis' theorem or
  Wick's theorem).
  %
  We also develop new tools for bounding smallest singular values of structured random
  matrices, which could be useful in other smoothed analysis settings.



\textbf{``Efficiency-Risk Tradeoffs in Electricity Markets with Dynamic Demand Response''}
\\
In order to study the impact of dynamic demand response in the future smart grid, we examine in an
abstract framework, how a tradeoff between efficiency and risk arises under different market
architectures.
\\
We first examine the system performance under non-cooperative and cooperative market architectures.
%
The statistics of the stationary aggregate demand processes show that, although the non-cooperative
load scheduling scheme leads to an efficiency loss, the stationary distribution of the corresponding
aggregate demand process has a smaller tail, resulting in less frequent aggregate demand spikes.
%
{Cooperative dynamic demand response, on the other hand, makes the market place more efficient at
  the cost of increased risk of aggregate demand spikes. The market architecture determines the
  locus of the system performance with respect to the tradeoff curve.}
\\
We also investigate how a properly designed real-time electricity pricing mechanism can help the
system operator achieve a target tradeoff between efficiency and risk in a non-cooperative market.
%
We further provide a convex characterization of the Pareto front of system performance measures,
which serves as a benchmark of the tradeoffs for the system operator to evaluate the pricing rules.


 % \begin{center}
 %   {\scriptsize  Last updated: \today\- •\- }
 % \end{center}

\end{resume}

\end{document}


%____________________________________________________________
% EOF
